\documentclass{article}
\usepackage[utf8]{inputenc}
\usepackage{natbib}
\bibliographystyle{unsrtnat}

\title{Scientific comments on the Master's Thesis Extension of the Progressive Mesh Representation for 3D Triangular Meshes by Roland Mattsson. Reviewer:
Hampus Fristedt.}
\author{Hampus Fristedt}

\begin{document}
\maketitle

\newpage

\section{Summary}

Polygon meshes are commonly used in computer graphics. The meshes are
collections of geometric figues (commonly triangles) that form 3D models. These
data structures are often large and can require great amounts of storage and
processing power. An alternative to poylgon meshes are progressive meshes.
Progressive meshes were developed at Microsoft Research by Hugues Hoppe in 1996
and offer flexible level of detail and loss-less compression. 

The representation is useful when bandwidth and processing power is limited. An
analogy can be be made to digital pictures. For example, a user visits a website
from their cellphone. The webserver detects that the user has poor bandwidth and
serves a low-resolution image. Another user visits the same website from their
desktop computer with a fiber connection. This user would instead be served a
high-resolution image. In a similiar fashion, a progressive mesh can be served
either in its "high-resolution" uncompressed form, or it can be served in a
compressed "low-resolution" format with fewer polygons. Note that while a
high-resolution image rarely can be restored from a low-resolution image, a
progressive mesh can always be restored to its original "high-resolution" state.

A movie is commonly shot and presented 24 frames each second. To achieve a
smooth look, computer games require even more frames per second (FPS), 60 FPS
being a common amount. Transmitting entire meshes every frame can be extremely
demanding of both processors and bandwidth, even when using progressive meshes.
Using articulated models is an efficient solution.

An articulated model consists of an ordinary 3D-model, e.g. a mesh, and a set of
parameters that control how the mesh appears. Instead of sending an entire new
mesh every frame of an animation, the mesh is sent only once and new parameters
are sent that represent the modifications of the model. This can drastically
reduce the amount of data transmitted.

Hoppe's description of progressive meshes did not describe how to combine his
representation with articulated models. Roland Mattsson's master thesis
describes how such a combination can be done. Mattson describes how animation of
the progressive mesh can be done at any level of detail as well as a framework
for controlling the level of detail of the mesh. Because both progressive meshes
and articulated models are useful in situations where performance is limited,
Mattsson spent extra effort in creating an efficient implementation, without
sacrificing useful features.

The articulated progressive meshes presented by Mattsson enable useful
techniques. For example, different parts of a model can be represented at
different levels of detail. A human model might be represented by a progressive
mesh that is uncompressed above the shoulders to allow for an expressive face,
but with a reduced number of polygons for the rest of the body. Objects far from
the camera might also be served with fewer polygons than close objects.

\newpage

\section{Critical review}

\subsection{Language and structure}

The paper is overall very well written. The language is coherent, academic yet
parseable. I did find one small spelling error ("correspons" on page 11).

The included reading instructions state that the text follows a bottom-up
approach, starting with an introduction to computer graphics and concluding with
the author's own extension of progressive meshes and a summary. The bottom-up
approach is applied well and the first sections of the report can be used by
someone uninitiated as an introduction to the subject. The chosen structure of
the report does however mean that the new, unique part of the work is placed at
the very end of the report (starts at page 39 of 51).

The abstract is brief, correct and informative. The reader is informed of the
topic of the thesis immediately and the rest of the abstract is used describing
the contents of the report. The title could have been more specific regarding
what type of extension that is implemented, i.e. mention both progressive
meshes and articulated models, instead of just the former.

The background section is very informative and briefly describes most subjects
within the scope of the work. The section could be used to give an uninformed
reader a very good idea of what the work is about, why it exists, and whether it
succeeded.

The background section of the report is followed by a quite lengthy introduction
to computer graphics that covers the theoretical basics of the subject. It
describes applications, algebraic models, animation techniques and rendering. On
pages 10 through 12, the author provides an almost needlessly extensive overview
of meshes. The author does however defend the section by stating that it
"constitutes a context for the progressive mesh representation". I question
whether the section on parametric representations of objects is relevant to the
report. Parametric representation is clearly an important part of computer
graphics, but the work is about polygonal meshes that by definition are not
parametric representations. The section about animation starts with an
accessible introduction and continues with a section about different
classification techniques for animation, and concludes by applying one of the
classification techniques on different types of animation. The final part of the
background is about rendering. Like the other parts, it is interesting but is of
questionable relevance to the the work.

Although I personally would prefer a more focused work with a shorter
introduction, one of the criteria of a master's degree project is to
"demonstrate knowledge of the chosen topic's disciplinary foundation and proven
experience, insight into current research and development, as well as in-depth
knowledge of methodology". Therefore, I cannot fault the author for including
such a lengthy and accessible introduction.

\subsection{Choice of topic}

The author chose to base his work on the report Efficient Implementation of
Progressive Meshes by \citet{hoppe1998}. This seems like a wise choice
as Hoppe introduced the concept \citep{hoppe}. The chosen extension is a natural
continuation of the research by Hoppe.

\subsection{Methodology}


\bibliography{references}

\end{document}
